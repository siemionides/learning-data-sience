\documentclass[]{article}
\usepackage{lmodern}
\usepackage{amssymb,amsmath}
\usepackage{ifxetex,ifluatex}
\usepackage{fixltx2e} % provides \textsubscript
\ifnum 0\ifxetex 1\fi\ifluatex 1\fi=0 % if pdftex
  \usepackage[T1]{fontenc}
  \usepackage[utf8]{inputenc}
\else % if luatex or xelatex
  \ifxetex
    \usepackage{mathspec}
  \else
    \usepackage{fontspec}
  \fi
  \defaultfontfeatures{Ligatures=TeX,Scale=MatchLowercase}
\fi
% use upquote if available, for straight quotes in verbatim environments
\IfFileExists{upquote.sty}{\usepackage{upquote}}{}
% use microtype if available
\IfFileExists{microtype.sty}{%
\usepackage{microtype}
\UseMicrotypeSet[protrusion]{basicmath} % disable protrusion for tt fonts
}{}
\usepackage[margin=1in]{geometry}
\usepackage{hyperref}
\hypersetup{unicode=true,
            pdftitle={Descriptive Statistics Report},
            pdfauthor={Michał Siemiończyk},
            pdfborder={0 0 0},
            breaklinks=true}
\urlstyle{same}  % don't use monospace font for urls
\usepackage{color}
\usepackage{fancyvrb}
\newcommand{\VerbBar}{|}
\newcommand{\VERB}{\Verb[commandchars=\\\{\}]}
\DefineVerbatimEnvironment{Highlighting}{Verbatim}{commandchars=\\\{\}}
% Add ',fontsize=\small' for more characters per line
\usepackage{framed}
\definecolor{shadecolor}{RGB}{248,248,248}
\newenvironment{Shaded}{\begin{snugshade}}{\end{snugshade}}
\newcommand{\KeywordTok}[1]{\textcolor[rgb]{0.13,0.29,0.53}{\textbf{{#1}}}}
\newcommand{\DataTypeTok}[1]{\textcolor[rgb]{0.13,0.29,0.53}{{#1}}}
\newcommand{\DecValTok}[1]{\textcolor[rgb]{0.00,0.00,0.81}{{#1}}}
\newcommand{\BaseNTok}[1]{\textcolor[rgb]{0.00,0.00,0.81}{{#1}}}
\newcommand{\FloatTok}[1]{\textcolor[rgb]{0.00,0.00,0.81}{{#1}}}
\newcommand{\ConstantTok}[1]{\textcolor[rgb]{0.00,0.00,0.00}{{#1}}}
\newcommand{\CharTok}[1]{\textcolor[rgb]{0.31,0.60,0.02}{{#1}}}
\newcommand{\SpecialCharTok}[1]{\textcolor[rgb]{0.00,0.00,0.00}{{#1}}}
\newcommand{\StringTok}[1]{\textcolor[rgb]{0.31,0.60,0.02}{{#1}}}
\newcommand{\VerbatimStringTok}[1]{\textcolor[rgb]{0.31,0.60,0.02}{{#1}}}
\newcommand{\SpecialStringTok}[1]{\textcolor[rgb]{0.31,0.60,0.02}{{#1}}}
\newcommand{\ImportTok}[1]{{#1}}
\newcommand{\CommentTok}[1]{\textcolor[rgb]{0.56,0.35,0.01}{\textit{{#1}}}}
\newcommand{\DocumentationTok}[1]{\textcolor[rgb]{0.56,0.35,0.01}{\textbf{\textit{{#1}}}}}
\newcommand{\AnnotationTok}[1]{\textcolor[rgb]{0.56,0.35,0.01}{\textbf{\textit{{#1}}}}}
\newcommand{\CommentVarTok}[1]{\textcolor[rgb]{0.56,0.35,0.01}{\textbf{\textit{{#1}}}}}
\newcommand{\OtherTok}[1]{\textcolor[rgb]{0.56,0.35,0.01}{{#1}}}
\newcommand{\FunctionTok}[1]{\textcolor[rgb]{0.00,0.00,0.00}{{#1}}}
\newcommand{\VariableTok}[1]{\textcolor[rgb]{0.00,0.00,0.00}{{#1}}}
\newcommand{\ControlFlowTok}[1]{\textcolor[rgb]{0.13,0.29,0.53}{\textbf{{#1}}}}
\newcommand{\OperatorTok}[1]{\textcolor[rgb]{0.81,0.36,0.00}{\textbf{{#1}}}}
\newcommand{\BuiltInTok}[1]{{#1}}
\newcommand{\ExtensionTok}[1]{{#1}}
\newcommand{\PreprocessorTok}[1]{\textcolor[rgb]{0.56,0.35,0.01}{\textit{{#1}}}}
\newcommand{\AttributeTok}[1]{\textcolor[rgb]{0.77,0.63,0.00}{{#1}}}
\newcommand{\RegionMarkerTok}[1]{{#1}}
\newcommand{\InformationTok}[1]{\textcolor[rgb]{0.56,0.35,0.01}{\textbf{\textit{{#1}}}}}
\newcommand{\WarningTok}[1]{\textcolor[rgb]{0.56,0.35,0.01}{\textbf{\textit{{#1}}}}}
\newcommand{\AlertTok}[1]{\textcolor[rgb]{0.94,0.16,0.16}{{#1}}}
\newcommand{\ErrorTok}[1]{\textcolor[rgb]{0.64,0.00,0.00}{\textbf{{#1}}}}
\newcommand{\NormalTok}[1]{{#1}}
\usepackage{graphicx,grffile}
\makeatletter
\def\maxwidth{\ifdim\Gin@nat@width>\linewidth\linewidth\else\Gin@nat@width\fi}
\def\maxheight{\ifdim\Gin@nat@height>\textheight\textheight\else\Gin@nat@height\fi}
\makeatother
% Scale images if necessary, so that they will not overflow the page
% margins by default, and it is still possible to overwrite the defaults
% using explicit options in \includegraphics[width, height, ...]{}
\setkeys{Gin}{width=\maxwidth,height=\maxheight,keepaspectratio}
\IfFileExists{parskip.sty}{%
\usepackage{parskip}
}{% else
\setlength{\parindent}{0pt}
\setlength{\parskip}{6pt plus 2pt minus 1pt}
}
\setlength{\emergencystretch}{3em}  % prevent overfull lines
\providecommand{\tightlist}{%
  \setlength{\itemsep}{0pt}\setlength{\parskip}{0pt}}
\setcounter{secnumdepth}{0}
% Redefines (sub)paragraphs to behave more like sections
\ifx\paragraph\undefined\else
\let\oldparagraph\paragraph
\renewcommand{\paragraph}[1]{\oldparagraph{#1}\mbox{}}
\fi
\ifx\subparagraph\undefined\else
\let\oldsubparagraph\subparagraph
\renewcommand{\subparagraph}[1]{\oldsubparagraph{#1}\mbox{}}
\fi

%%% Use protect on footnotes to avoid problems with footnotes in titles
\let\rmarkdownfootnote\footnote%
\def\footnote{\protect\rmarkdownfootnote}

%%% Change title format to be more compact
\usepackage{titling}

% Create subtitle command for use in maketitle
\newcommand{\subtitle}[1]{
  \posttitle{
    \begin{center}\large#1\end{center}
    }
}

\setlength{\droptitle}{-2em}
  \title{Descriptive Statistics Report}
  \pretitle{\vspace{\droptitle}\centering\huge}
  \posttitle{\par}
  \author{Michał Siemiończyk}
  \preauthor{\centering\large\emph}
  \postauthor{\par}
  \predate{\centering\large\emph}
  \postdate{\par}
  \date{10 marca 2017}


\begin{document}
\maketitle

{
\setcounter{tocdepth}{2}
\tableofcontents
}
\section{Introduction}\label{introduction}

The task was to analyze the W1\_OcenyIlorazIntelZarobki.xlsx dataset.

\begin{itemize}
\tightlist
\item
  for each variable there was a descriptive (summary) analisys to be
  performed
\item
  the modality of variables distribution has to be checked
\item
  check if the variables have normal distribution
\item
  draw histograms and boxplots and analize its distribution
\item
  check for outstanding values (outliers)
\item
  if there's a grouping variable
\item
  perform the task for each of them
\end{itemize}

(in Polish:Przeanalizuj załączone zbiory. Dla każdej zmiennej oblicz
statystyki opisowe sprawdź modalność rozkładów sprawdź czy zmienne mają
rozkład normalny narysuj histogramy narysuj wykresy ``skrzynka z
wąsami'' i przeanalizuj rozkład, występowanie obserwacji odstających i
ekstremalnych Jeśli w danych występuje zmienna grupująca powtórz zadanie
w każdej z grup. )

\begin{Shaded}
\begin{Highlighting}[]
\NormalTok{data =}\StringTok{ }\KeywordTok{read.xls}\NormalTok{(}\StringTok{'W1_OcenyIlorazIntelZarobki.xlsx'}\NormalTok{)}
\end{Highlighting}
\end{Shaded}

\section{Descriptive analisys}\label{descriptive-analisys}

The function \texttt{opisowe\_statystyki} from \texttt{functions.R} was
used to perform simple descriptive analisys.

\begin{Shaded}
\begin{Highlighting}[]
\KeywordTok{source}\NormalTok{(}\StringTok{"../functions.R"}\NormalTok{)}
\end{Highlighting}
\end{Shaded}

\subsection{\texorpdfstring{Variable
``IQ''}{Variable IQ}}\label{variable-iq}

\subsubsection{All IQ Data}\label{all-iq-data}

\begin{Shaded}
\begin{Highlighting}[]
\NormalTok{iqData =}\StringTok{ }\NormalTok{data$iloraz_intelig}
\end{Highlighting}
\end{Shaded}

\begin{Shaded}
\begin{Highlighting}[]
\KeywordTok{summary}\NormalTok{(iqData)}
\end{Highlighting}
\end{Shaded}

\begin{verbatim}
   Min. 1st Qu.  Median    Mean 3rd Qu.    Max. 
  67.00   86.25   97.00   97.01  107.20  134.00 
\end{verbatim}

\begin{Shaded}
\begin{Highlighting}[]
\KeywordTok{descriptive_statistics}\NormalTok{(iqData, }\StringTok{"IQ variable"}\NormalTok{)}
\end{Highlighting}
\end{Shaded}

\begin{verbatim}
variance 194.171616161616
\end{verbatim}

\begin{verbatim}
standard dev 13.9345475764955
\end{verbatim}

\begin{verbatim}
skewness 0.128685339310997 (jeżeli dodatnie to prawy brzeg jest dłuższy, jeżeli ujemne - to lewy)
\end{verbatim}

\begin{verbatim}
kurtoza -0.606481574472046 + : wystaje ponad normalny; - : poniżej normalnego
\end{verbatim}

\begin{verbatim}
shapiro: p-value:  0.432193565052699 (if p-value < 0.05 - NOT NORMAL)
\end{verbatim}

\includegraphics{raport_files/figure-latex/unnamed-chunk-5-1.pdf}

All samples for variable IQ result int P-value greater than \(\alpha\) =
0.05 - therefore they have a normal distribution.

\subsubsection{Group A}\label{group-a}

\begin{Shaded}
\begin{Highlighting}[]
\NormalTok{iqGroupAData =}\StringTok{ }\NormalTok{data$iloraz_intelig[data$grupa==}\StringTok{'A'}\NormalTok{]}
\end{Highlighting}
\end{Shaded}

\begin{Shaded}
\begin{Highlighting}[]
\KeywordTok{summary}\NormalTok{(iqGroupAData)}
\end{Highlighting}
\end{Shaded}

\begin{verbatim}
   Min. 1st Qu.  Median    Mean 3rd Qu.    Max. 
   80.0   100.5   106.5   104.9   111.5   122.0 
\end{verbatim}

\begin{Shaded}
\begin{Highlighting}[]
\KeywordTok{descriptive_statistics}\NormalTok{(iqGroupAData, }\StringTok{"IQ variable groupped for Group A"}\NormalTok{)}
\end{Highlighting}
\end{Shaded}

\begin{verbatim}
variance 154.696428571429
\end{verbatim}

\begin{verbatim}
standard dev 12.4377019007302
\end{verbatim}

\begin{verbatim}
skewness -0.636340940511677 (jeżeli dodatnie to prawy brzeg jest dłuższy, jeżeli ujemne - to lewy)
\end{verbatim}

\begin{verbatim}
kurtoza -0.511865761392951 + : wystaje ponad normalny; - : poniżej normalnego
\end{verbatim}

\begin{verbatim}
shapiro: p-value:  0.593628360075638 (if p-value < 0.05 - NOT NORMAL)
\end{verbatim}

\includegraphics{raport_files/figure-latex/unnamed-chunk-8-1.pdf}

\subsubsection{Group B}\label{group-b}

\begin{Shaded}
\begin{Highlighting}[]
\NormalTok{iqGroupBData =}\StringTok{ }\NormalTok{data$iloraz_intelig[data$grupa==}\StringTok{'B'}\NormalTok{]}
\end{Highlighting}
\end{Shaded}

\begin{Shaded}
\begin{Highlighting}[]
\KeywordTok{summary}\NormalTok{(iqGroupBData)}
\end{Highlighting}
\end{Shaded}

\begin{verbatim}
   Min. 1st Qu.  Median    Mean 3rd Qu.    Max. 
  72.00   93.00  100.50   98.25  105.50  109.00 
\end{verbatim}

\begin{Shaded}
\begin{Highlighting}[]
\KeywordTok{descriptive_statistics}\NormalTok{(iqGroupBData, }\StringTok{"IQ variable groupped for Group B"}\NormalTok{)}
\end{Highlighting}
\end{Shaded}

\begin{verbatim}
variance 109.113636363636
\end{verbatim}

\begin{verbatim}
standard dev 10.4457472860316
\end{verbatim}

\begin{verbatim}
skewness -1.15968767806069 (jeżeli dodatnie to prawy brzeg jest dłuższy, jeżeli ujemne - to lewy)
\end{verbatim}

\begin{verbatim}
kurtoza 0.602479849687347 + : wystaje ponad normalny; - : poniżej normalnego
\end{verbatim}

\begin{verbatim}
shapiro: p-value:  0.056787250779079 (if p-value < 0.05 - NOT NORMAL)
\end{verbatim}

\includegraphics{raport_files/figure-latex/unnamed-chunk-11-1.pdf}

\subsection{Grades}\label{grades}

\begin{Shaded}
\begin{Highlighting}[]
\NormalTok{grades =}\StringTok{ }\NormalTok{data$oceny}
\NormalTok{grades =}\StringTok{ }\NormalTok{grades[!}\KeywordTok{is.na}\NormalTok{(grades)]}
\end{Highlighting}
\end{Shaded}

\subsubsection{Ungroupped data}\label{ungroupped-data}

\begin{Shaded}
\begin{Highlighting}[]
\KeywordTok{summary}\NormalTok{(grades)}
\end{Highlighting}
\end{Shaded}

\begin{verbatim}
   Min. 1st Qu.  Median    Mean 3rd Qu.    Max. 
  2.000   3.000   3.500   3.525   4.000   5.000 
\end{verbatim}

\begin{Shaded}
\begin{Highlighting}[]
\KeywordTok{descriptive_statistics}\NormalTok{(grades, }\StringTok{"Grades variable for all samples"}\NormalTok{)}
\end{Highlighting}
\end{Shaded}

\begin{verbatim}
variance 0.801973684210526
\end{verbatim}

\begin{verbatim}
standard dev 0.89552983434977
\end{verbatim}

\begin{verbatim}
skewness -0.17531017067566 (jeżeli dodatnie to prawy brzeg jest dłuższy, jeżeli ujemne - to lewy)
\end{verbatim}

\begin{verbatim}
kurtoza -0.822696832218342 + : wystaje ponad normalny; - : poniżej normalnego
\end{verbatim}

\begin{verbatim}
shapiro: p-value:  0.200116183030953 (if p-value < 0.05 - NOT NORMAL)
\end{verbatim}

\includegraphics{raport_files/figure-latex/unnamed-chunk-14-1.pdf}

\subsubsection{Group A}\label{group-a-1}

\begin{Shaded}
\begin{Highlighting}[]
\NormalTok{gradesGroupA =}\StringTok{ }\NormalTok{data$oceny[data$grupa==}\StringTok{'A'}\NormalTok{]}
\NormalTok{gradesGroupA =}\StringTok{ }\NormalTok{gradesGroupA[!}\KeywordTok{is.na}\NormalTok{(gradesGroupA)]}
\end{Highlighting}
\end{Shaded}

\begin{Shaded}
\begin{Highlighting}[]
\KeywordTok{summary}\NormalTok{(gradesGroupA)}
\end{Highlighting}
\end{Shaded}

\begin{verbatim}
   Min. 1st Qu.  Median    Mean 3rd Qu.    Max. 
  2.000   2.750   3.250   3.312   4.000   5.000 
\end{verbatim}

\begin{Shaded}
\begin{Highlighting}[]
\KeywordTok{descriptive_statistics}\NormalTok{(gradesGroupA, }\StringTok{"Grades - Group  A"}\NormalTok{)}
\end{Highlighting}
\end{Shaded}

\begin{verbatim}
variance 1.06696428571429
\end{verbatim}

\begin{verbatim}
standard dev 1.0329396331414
\end{verbatim}

\begin{verbatim}
skewness 0.0996845245652554 (jeżeli dodatnie to prawy brzeg jest dłuższy, jeżeli ujemne - to lewy)
\end{verbatim}

\begin{verbatim}
kurtoza -1.40662447917578 + : wystaje ponad normalny; - : poniżej normalnego
\end{verbatim}

\begin{verbatim}
shapiro: p-value:  0.621541925436456 (if p-value < 0.05 - NOT NORMAL)
\end{verbatim}

\includegraphics{raport_files/figure-latex/unnamed-chunk-17-1.pdf}

\subsubsection{Group B}\label{group-b-1}

\begin{Shaded}
\begin{Highlighting}[]
\NormalTok{gradesGroupB =}\StringTok{ }\NormalTok{data$oceny[data$grupa==}\StringTok{'B'}\NormalTok{]}
\NormalTok{gradesGroupB =}\StringTok{ }\NormalTok{gradesGroupB[!}\KeywordTok{is.na}\NormalTok{(gradesGroupB)]}
\end{Highlighting}
\end{Shaded}

\begin{Shaded}
\begin{Highlighting}[]
\KeywordTok{summary}\NormalTok{(gradesGroupB)}
\end{Highlighting}
\end{Shaded}

\begin{verbatim}
   Min. 1st Qu.  Median    Mean 3rd Qu.    Max. 
  2.000   3.375   3.500   3.667   4.125   5.000 
\end{verbatim}

\begin{Shaded}
\begin{Highlighting}[]
\KeywordTok{descriptive_statistics}\NormalTok{(gradesGroupB, }\StringTok{"Grades - Group  B"}\NormalTok{)}
\end{Highlighting}
\end{Shaded}

\begin{verbatim}
variance 0.651515151515151
\end{verbatim}

\begin{verbatim}
standard dev 0.807164884961649
\end{verbatim}

\begin{verbatim}
skewness -0.259705701899279 (jeżeli dodatnie to prawy brzeg jest dłuższy, jeżeli ujemne - to lewy)
\end{verbatim}

\begin{verbatim}
kurtoza -0.592324680007211 + : wystaje ponad normalny; - : poniżej normalnego
\end{verbatim}

\begin{verbatim}
shapiro: p-value:  0.737556857059956 (if p-value < 0.05 - NOT NORMAL)
\end{verbatim}

\includegraphics{raport_files/figure-latex/unnamed-chunk-20-1.pdf}

\subsection{Income}\label{income}

\begin{Shaded}
\begin{Highlighting}[]
\NormalTok{income =}\StringTok{ }\NormalTok{data$zarobki}
\NormalTok{income =}\StringTok{ }\NormalTok{income[!}\KeywordTok{is.na}\NormalTok{(income)]}
\CommentTok{#income}
\end{Highlighting}
\end{Shaded}

\subsubsection{Ungroupped data}\label{ungroupped-data-1}

\begin{Shaded}
\begin{Highlighting}[]
\KeywordTok{summary}\NormalTok{(income)}
\end{Highlighting}
\end{Shaded}

\begin{verbatim}
   Min. 1st Qu.  Median    Mean 3rd Qu.    Max. 
    600    1500    1800    3718    2200   40000 
\end{verbatim}

\begin{Shaded}
\begin{Highlighting}[]
\KeywordTok{descriptive_statistics}\NormalTok{(income, }\StringTok{"Income variable for all samples"}\NormalTok{)}
\end{Highlighting}
\end{Shaded}

\begin{verbatim}
variance 61636777.3333333
\end{verbatim}

\begin{verbatim}
standard dev 7850.90933162098
\end{verbatim}

\begin{verbatim}
skewness 3.9776498610461 (jeżeli dodatnie to prawy brzeg jest dłuższy, jeżeli ujemne - to lewy)
\end{verbatim}

\begin{verbatim}
kurtoza 15.3005986972439 + : wystaje ponad normalny; - : poniżej normalnego
\end{verbatim}

\begin{verbatim}
shapiro: p-value:  1.25005130097941e-09 (if p-value < 0.05 - NOT NORMAL)
\end{verbatim}

\includegraphics{raport_files/figure-latex/unnamed-chunk-23-1.pdf}

\subsubsection{Group A}\label{group-a-2}

\begin{Shaded}
\begin{Highlighting}[]
\NormalTok{incomeGroupA =}\StringTok{ }\NormalTok{data$zarobki[data$grupa==}\StringTok{'A'}\NormalTok{]}
\NormalTok{incomeGroupA =}\StringTok{ }\NormalTok{incomeGroupA[!}\KeywordTok{is.na}\NormalTok{(incomeGroupA)]}

\KeywordTok{summary}\NormalTok{(incomeGroupA)}
\end{Highlighting}
\end{Shaded}

\begin{verbatim}
   Min. 1st Qu.  Median    Mean 3rd Qu.    Max. 
    600    1500    1890    1898    2075    3500 
\end{verbatim}

\begin{Shaded}
\begin{Highlighting}[]
\KeywordTok{descriptive_statistics}\NormalTok{(incomeGroupA, }\StringTok{"Income - Group A"}\NormalTok{)}
\end{Highlighting}
\end{Shaded}

\begin{verbatim}
variance 679478.571428571
\end{verbatim}

\begin{verbatim}
standard dev 824.30490198019
\end{verbatim}

\begin{verbatim}
skewness 0.417602447001236 (jeżeli dodatnie to prawy brzeg jest dłuższy, jeżeli ujemne - to lewy)
\end{verbatim}

\begin{verbatim}
kurtoza -0.426506097263498 + : wystaje ponad normalny; - : poniżej normalnego
\end{verbatim}

\begin{verbatim}
shapiro: p-value:  0.534382538045862 (if p-value < 0.05 - NOT NORMAL)
\end{verbatim}

\includegraphics{raport_files/figure-latex/unnamed-chunk-25-1.pdf}

\subsubsection{Group B}\label{group-b-2}

\begin{Shaded}
\begin{Highlighting}[]
\NormalTok{incomeGroupB =}\StringTok{ }\NormalTok{data$zarobki[data$grupa==}\StringTok{'B'}\NormalTok{]}
\NormalTok{incomeGroupB =}\StringTok{ }\NormalTok{incomeGroupB[!}\KeywordTok{is.na}\NormalTok{(incomeGroupB)]}

\KeywordTok{summary}\NormalTok{(incomeGroupB)}
\end{Highlighting}
\end{Shaded}

\begin{verbatim}
   Min. 1st Qu.  Median    Mean 3rd Qu.    Max. 
   1250    1692    2050    2756    2200   12000 
\end{verbatim}

\begin{Shaded}
\begin{Highlighting}[]
\KeywordTok{descriptive_statistics}\NormalTok{(incomeGroupB, }\StringTok{"Income - Group B"}\NormalTok{)}
\end{Highlighting}
\end{Shaded}

\begin{verbatim}
variance 8582590.15151515
\end{verbatim}

\begin{verbatim}
standard dev 2929.60580138611
\end{verbatim}

\begin{verbatim}
skewness 2.58614295802334 (jeżeli dodatnie to prawy brzeg jest dłuższy, jeżeli ujemne - to lewy)
\end{verbatim}

\begin{verbatim}
kurtoza 5.27416159472823 + : wystaje ponad normalny; - : poniżej normalnego
\end{verbatim}

\begin{verbatim}
shapiro: p-value:  6.01253589596165e-06 (if p-value < 0.05 - NOT NORMAL)
\end{verbatim}

\includegraphics{raport_files/figure-latex/unnamed-chunk-27-1.pdf}


\end{document}
